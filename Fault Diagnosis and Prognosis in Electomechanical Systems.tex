\documentclass{article}
\usepackage{amsmath,amssymb}
\usepackage[latin1]{inputenc}
\usepackage[francais]{babel}
\usepackage{tikz,pgfplots}           %for TikZ graphics
\pgfplotsset{compat=1.9}
\usepackage{caption}
\usepackage{subcaption}

\usepgfplotslibrary{fillbetween}
\begin{document}

\title{Advanced Signal Processing Techniques for Electrical Signal Monitoring}
   
 \maketitle  
 
The energy transition is one of the most important societal challenge of the next decades. This transition relies on a massive increase of renewable sources (wind, solar, ...) in the energy mix and on the introduction of new usages (EV, HEV). Despite its ecological benefits, the energy transition has also raised new technical challenges. First, the balance between energy consumption and production becomes more difficult to maintain due to the use of intermittent energy sources. Then, the power reliability and quality are critically affected by 


%Introduction
Fault detection and diagnosis
% Travaux

%Limitations
Despite its statistical performance, Parametric approaches for fault detection have several drawbacks. First, the statistical performance drastically degrade in presence of model mismatch (wrong number of spectral components, non-gaussian noise, presence of additional frequency componenents). Then, these database-free techniques are suboptimal when a signal database is available since they do not exploit the extra information provided by any previous recorded measurements.
Last but not least, these techniques only focus on the estimation-detection steps. In particular, they does not provide any information about the remaining lifespan of the entire mechanical system and do not provide any recommendation for the maintenance plannification.

%Technique
One objective of this thesis is to investigate the use of advanced signal processing and patter recognition techniques for the diagnosis and prognosis of electromechanical systems. Specifically, several research area can be investigated.

\begin{itemize}
\item Sparse representation for spectral analysis. Conventional techniques for spectral analysis (or Direction of Arrivals) include non-parametric (periodogram) and parametric approaches (Maximum Likelihood, subspace techniques). While non-parametric approaches suffer from low statistical performances, the parametric approaches are highly sensitive to model mismatch. To overcome these limitations, several authors have recently proposed to use semi-parametric techniques based on sparse representations. Technically, sparse representations exploit the fact the signal spectrum contains a large number of null components. These techniques offers better statistical performance than non-parametric approach while preserving high robustness. In the context of system diagnosis, it will be interesting to investigate the advantages of this cutting edge technique over conventional approaches with electrical and vibrations signals. 

\item Deep Learning for Diagnosis and Prognostic. When a database is available, the diagnosis and the prognostic are usually performed using pattern recognition approaches such as Neural Networks or Support Vector Machines. However, it is well known that the performance of these techniques critically depend on the feature extraction stage. In practice, a good choice of features is a difficult engineering task and requires appropritate skills in signal processing and statistics. Recently, new neural architecture based on Deep Learning have proposed to overcome this problem by directly sending the raw signal to the deep network inputs. In this context, it will be interesting to investigate the benefit of this new approach for the diagnosis and prognosis of the system states.
\end{itemize}


@article{malioutov2005sparse,
  title={A sparse signal reconstruction perspective for source localization with sensor arrays},
  author={Malioutov, Dmitry and {\c{C}}etin, M{\"u}jdat and Willsky, Alan S},
  journal={IEEE transactions on signal processing},
  volume={53},
  number={8},
  pages={3010--3022},
  year={2005},
  publisher={IEEE
  
 }


@article{yang2016sparse,
  title={Sparse Methods for Direction-of-Arrival Estimation},
  author={Yang, Zai and Li, Jian and Stoica, Petre and Xie, Lihua},
  journal={arXiv preprint arXiv:1609.09596},
  year={2016}
}

@inproceedings{stoica2014gridless,
  title={Gridless compressive-sensing methods for frequency estimation: Points of tangency and links to basics},
  author={Stoica, Petre and Tangy, Gongguo and Yangz, Zai and Zachariah, Dave},
  booktitle={Signal Processing Conference (EUSIPCO), 2014 Proceedings of the 22nd European},
  pages={1831--1835},
  year={2014},
  organization={IEEE}
}

   
\end{document}
