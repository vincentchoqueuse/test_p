\documentclass{article}
\usepackage{amsmath,amssymb}
\usepackage[latin1]{inputenc}
\usepackage[francais]{babel}
\usepackage{tikz,pgfplots}           %for TikZ graphics
\pgfplotsset{compat=1.9}
\usepackage{caption}
\usepackage{subcaption}
\usepackage[left=2cm,right=2cm,top=2cm,bottom=2cm]{geometry}
\usepgfplotslibrary{fillbetween}
\begin{document}

\title{Advanced Signal Processing Techniques for Electrical Signal Monitoring}
   
 \maketitle  
 

\section{Context}
%Introduction

Faults diagnosis and prognosis in electromecanical systems is of paramount importance in order to improve rotating machines reliability and availability and reduce operating and maintenance costs. Condition based maintenance of rotating machines in industrial applications is based on performance and physical parameters monitoring (vibration, currents, temperature, lubrication). The existing technologies for electrical machines condition monitoring are vibration monitoring, torque monitoring, temperature monitoring, and oil/debris monitoring. These technologies require additional sensors and specific data acquisition devices to be implemented. Prognosis is based on the estimation of the remaining operational life and the probability of failure of the system based on the acquired condition monitoring data. An increasing number of publications have focused on rotating machinery diagnostic and prognostic in the last few years. Many previous works have proposed to use advanced signal processing techniques as a medium for faults detection and diagnosis. Most of these techniques are based on power spectral density (PSD) estimation or time frequency representations. These signal processing techniques allow extracting useful indicators of the fault from raw data. Regarding the decision making, the existing approaches can be classified in two main categories which are the detection theory and artificial intelligence (AI) techniques (Artificial Neural Networks,  Support Vector Machines, fuzzy logic).

% Travaux

\section{Previous works}

The IRDL Laboratory and the Energy and Electromechanical Systems (EES) team of ISEN Brest collaboration has allowed developing innovative faults diagnosis techniques in induction machines through stator currents analysis \cite{CHO12,TRA16,TRA17,ELB16}. As compared to other exiting approaches, the proposed techniques primarily focus on parametric analysis and offer better estimator and detector performances by taking profit of the particular signal structure. However, despite their statistical performance, parametric approaches have several drawbacks. First, it is known that the statistical performance drastically degrade in presence of model mismatch (wrong number of spectral components, non-gaussian noise, presence of additional frequency components). Then, these techniques are database-free and are inherently suboptimal when a signal database is available since they do not exploit the extra information provided by additional ressource.
Last but not least, these techniques only focus on the estimation-detection steps. In particular, they does not provide any information about the remaining lifespan of the entire mechanical system and do not provide any recommendation for the maintenance planification.

\section{Methodology}
The objective of this thesis is to overcome the limitations of parametric approaches for the monitoring of electromechanical systems. Specifically, alternative techniques based on advanced signal processing and pattern recognition algorithms can offer new opportunities for the automatic diagnosis and prognosis of electromechanical systems.

\begin{itemize}
\item Sparse representation for spectral analysis. Conventional techniques for spectral analysis (or Direction of Arrivals) include non-parametric (periodogram) and parametric approaches (Maximum Likelihood, subspace techniques). While non-parametric approaches suffer from low statistical performances, the parametric approaches are highly sensitive to model mismatch. To overcome these limitations, several authors have recently proposed to use semi-parametric techniques based on sparse representations  \cite{BHA13,TAN13,YAN16}. Technically, sparse representations exploit the fact that the signal spectrum contains a large number of null components. These representations offers better statistical performance than non-parametric approaches while preserving high robustness against non-gaussian noise and the presence of additional components. In the context of system diagnosis, it will be interesting to investigate the advantages of this cutting edge technique over conventional approaches for the spectral estimation of electrical and vibrations signals. 

\item Deep Learning for Diagnosis and Prognostic. When a database is available, the diagnosis and the prognostic are usually performed by pattern recognition approaches such as Neural Networks or Support Vector Machines \cite{BIS06}. However, it is known that the performance of these techniques critically depend on the feature extraction stage. In practice, a good choice of features is a difficult engineering task and requires appropriate skills in signal processing and statistics. Recently, new neural architecture based on Deep Learning have proposed to overcome this problem by delegating the feature extraction stage to the network inputs. Deep neural networks have recently brought about breakthroughs in processing images, video, speech and audio signals~\cite{LECUN15}. In this context, it is also natural to expect interesting results for diagnosis and prognosis applications.

\end{itemize}

\bibliographystyle{plain}
\bibliography{biblio}
   
 
\end{document}
